%%%%%%%%%%%%%%%%%%%%%%%%%%%%%%%%%%%%%%%%%%%%%%%%%%%%%%%%%%%%%%%%%%%%%%
%
\section{Complex Structures}
%
%%%%%%%%%%%%%%%%%%%%%%%%%%%%%%%%%%%%%%%%%%%%%%%%%%%%%%%%%%%%%%%%%%%%%%

Here you can find how to use labeling, cross-references, footnotes,
etc. Some of them are standard \LaTeX\ commands,
described in this section.
Commands available only in the IVT environment
will be explained in the next section.

%%%%%%%%%%%%%%%%%%%%%%%%%%%%%%%%%%%%%%%%%%%%%%%%%%%%%%%%%%%%%%%%%%%%%%
\subsection{Labels and Cross-Refs}
\label{sec:compStructs-LabelsCrossRefs}
%%%%%%%%%%%%%%%%%%%%%%%%%%%%%%%%%%%%%%%%%%%%%%%%%%%%%%%%%%%%%%%%%%%%%%

Anywhere you want you can add a label (see above for this subsection).
Wherever you want you can refer (cross-ref) to this label by using the
``\textbackslash{}ref'' command.
Example:
Labels and Cross-Refs can be found in
\cref{sec:compStructs-LabelsCrossRefs}.

Labels can also be used in lists:

\begin{enumerate}
  \item\label{item:stepOne} Step one
  \item Step two
  \item Feedback (Goto~\ref{item:stepOne})
\end{enumerate}

Labels are very important in Figures and Tables. It is used to refer
to them
in the written text (see \cref{sec:compStructs-Figures,sec:compStructs-Tables}).

%%%%%%%%%%%%%%%%%%%%%%%%%%%%%%%%%%%%%%%%%%%%%%%%%%%%%%%%%%%%%%%%%%%%%%
\subsection{Clever referencing}\label{sec:CleverReferencing}
%%%%%%%%%%%%%%%%%%%%%%%%%%%%%%%%%%%%%%%%%%%%%%%%%%%%%%%%%%%%%%%%%%%%%%

The ``cleveref'' package provides the ``\textbackslash{}cref'' command to conveniently
reference any type of object\footnote{See
\href{http://www.ctan.org/tex-archive/help/Catalogue/entries/cleveref.html}{its CTAN entry},
accessed on May 28th, 2010.}. For example:

\begin{itemize}
  \item This is a reference to a single section: Labels are described
in \cref{sec:compStructs-LabelsCrossRefs}.
  \item This is a reference to multiple sections: This document
contains, among others,
\cref{sec:compStructs-LabelsCrossRefs,sec:footnotes,sec:CleverReferencing}.
  \item This is a reference to multiple subsequent sections: This
document contains, among others,
\cref{sec:compStructs-LabelsCrossRefs,sec:footnotes,sec:compStructs-Figures}.
  \item One can summarize refs to different types of objects. This
becomes clear when taking a look at this sentence with references to
\cref{sec:compStructs-LabelsCrossRefs,tab:labelOfTheTable}.
  \item \Cref{fig:labelOfTheSingleFigure}---this is how a reference to a figure
    should look like at the beginning of a sentence.
\end{itemize}

Note how ``cleveref'' automatically adds the object type. One does not
have to write it anymore.
But remember using the ``\textbackslash{}Cref'' command instead
at the beginning of the sentence!

%%%%%%%%%%%%%%%%%%%%%%%%%%%%%%%%%%%%%%%%%%%%%%%%%%%%%%%%%%%%%%%%%%%%%%
\subsection{Footnotes}\label{sec:footnotes}
%%%%%%%%%%%%%%%%%%%%%%%%%%%%%%%%%%%%%%%%%%%%%%%%%%%%%%%%%%%%%%%%%%%%%%

Footnotes are directly embedded in the text where you want to refer to
them.
Depending on the layout, they will appear at the bottom of the current
page or
at the end of the Section. Example:

This is a text\footnote{more information about ``text''} with a
footnote\footnote{more information about ``footnote''}.


%%%%%%%%%%%%%%%%%%%%%%%%%%%%%%%%%%%%%%%%%%%%%%%%%%%%%%%%%%%%%%%%%%%%%%
\subsection{Formulas}
%%%%%%%%%%%%%%%%%%%%%%%%%%%%%%%%%%%%%%%%%%%%%%%%%%%%%%%%%%%%%%%%%%%%%%

To write formulas in \LaTeX{} you can describe it as plain text.
You can either embed a formula into the text or add it on a separate
line.
In the second version formula numbers will be automatically added.
With labels you are able to refer to the formulas.

\subsubsection{Formula embedded in text}

The formula part is enclosed by \$. Examples:

bla bla bla $[-30min,+30min]$ bla bla bla.

bla bla bla bla bla bla $P > N$ bla bla bla $S_{j}$ bla $j$ bla bla
bla  $\beta$ bla bla bla.

\subsubsection{Formula as separate line}

Using the ``linenomath'' and ``equation'' environments the formulas will be placed on a
separate line with a number. Examples:

\begin{linenomath}   
\begin{equation}
  \label{eq:score-averaging}
  S = (1 - \alpha) \cdot S_\text{old} + \alpha \cdot S_\text{new} 
  \end{equation}
\end{linenomath}

\begin{linenomath}   
  \begin{equation}
  \label{eq:utility-total}
  U_\text{total} = \sum_{i=1}^{n} U_{\text{perf},i} + \sum_{i=1}^{n} U_{\text{late},i}
              + \sum_{i=1}^{n} U_{\text{travel},i}\,,
   \end{equation} 
\end{linenomath}

\begin{linenomath}   
    \begin{equation}   
    \label{eq:utility-perform}
  U_{\text{perf},i}(t_{\text{perf},i}) = \max \left[ 0 , \beta_{\text{perf}} \cdot
t^{*}_{i} \cdot
                \ln \left( \frac{t_{\text{perf},i}}{t_{0,i}} \right)
\right]\,,
   \end{equation} 
\end{linenomath}

\begin{linenomath}   
    \begin{equation}   
    \label{eq:typical-duration}
  t_{0,i} = t^{*}_{i} \cdot e^{-\zeta / (p \cdot t^{*}_i)}\,,
   \end{equation} 
\end{linenomath}

\begin{linenomath}   
    \begin{equation}   
    \label{eq:utility-perform-resolved}
  U_{\text{perf},i}(t_{\text{perf},i}) = 
      \max\left[ 0 , \beta_\text{perf} \cdot t^{*}_{i} \cdot
      \left( \ln\left( \frac{t_{\text{perf},i}}{t^{*}_{i}} \right) +
      \frac{\zeta}{p \cdot t^{*}_i} \right) \right].
   \end{equation} 
  \end{linenomath}

\begin{linenomath}   
    \begin{equation}   
    \label{eq:utility-late}
  U_{\text{late},i} = \begin{cases} 
    \beta_{\text{late}} \cdot t_{\text{late},i}
      &: t_{\text{late},i} \geq 0 \\
    0
      &: \text{otherwise}
  \end{cases}
   \end{equation} 
 \end{linenomath}

\begin{linenomath}   
    \begin{equation}   
    \label{eq:utility-travel}
  \begin{aligned}
    U_{\text{travel},i} &= \beta_{\text{travel}} \cdot t_{\text{travel},i} \\
    &= \beta_{\text{travel}} \cdot \ldots 
  \end{aligned}
   \end{equation} 
\end{linenomath}

\begin{flushleft}
Sometimes the \textit{flushleft} environment needs to be used after the equation with alignments. 
\end{flushleft}

With the labels you can refer to an equation with the ``ref'' command.
Examples:

As shown in \cref{eq:utility-total}, the total utility is\ldots{}
See \cref{eq:utility-travel} for the definition of the
travel-utility.

\subsubsection{Text mode}

Text in math mode looks ugly, and spaces are ignored:
$bafflingly ugly$.
Anything that consists of more than one letter
should be surrounded by the ``\textbackslash{}text'' command:
$\text{bafflingly ugly}$.
If you need italics, use ``\textbackslash{}text\{\textbackslash{}emph\{\ldots\}\}''
or simply
 ``\textbackslash{}mathit'';
the latter still eats spaces:
$\mathit{bafflingly ugly}$.

%%%%%%%%%%%%%%%%%%%%%%%%%%%%%%%%%%%%%%%%%%%%%%%%%%%%%%%%%%%%%%%%%%%%%%
\subsection{Citations and References}
%%%%%%%%%%%%%%%%%%%%%%%%%%%%%%%%%%%%%%%%%%%%%%%%%%%%%%%%%%%%%%%%%%%%%%

By using the IVT bibliography database you can refer to a reference by
using the command ``\textbackslash{}citep'' (citation in parentheses)
or ``\textbackslash{}citet''
(citation embedded in the sentence). With the unique key of the
bib-entry you will automatically refer to that reference and the
reference will automatically be added to the reference list. The way
the references will be sorted depends on the layout you use. 

Examples for ``\textbackslash{}citet'':

bla bla 
\citet{matsimbook} and \citet{axhausen2008income} 
bla bla bla bla bla bla bla bla bla bla
\citet{axhausen2008income,matsimbook}
bla bla bla bla bla bla bla bla.

Examples for ``\textbackslash{}citep'':

bla bla bla bla bla bla
\citep[e.g.,]{matsimbook,axhausen2008income}.

bla bla bla bla bla bla \citep[see also][pp.325-378]{axhausen2008income}.

bla bla bla bla bla bla \citep{axhausen2008income}.

Just like with the clever referencing commands (cf.~\cref{sec:CleverReferencing})
you should use ``\textbackslash{}Citet'' or ``\textbackslash{}Citep'' instead
at the beginning of a sentence.
\Citet{axhausen2008income}
would be otherwise shown as \citet{axhausen2008income}
in your paper.
